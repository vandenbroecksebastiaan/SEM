\documentclass[11pt]{article}
\usepackage{geometry}
\usepackage{tcolorbox}
\usepackage{hyperref}
\usepackage{microtype}
\usepackage{XCharter}
\usepackage{rotating}
\usepackage[backend=biber,sorting=none,style=apa]{biblatex}
\addbibresource{library.bib}
\geometry{
    a4paper,
    total={170mm,257mm},
    left=20mm,
    top=20mm,
}
\setlength{\parskip}{5pt}
\setlength\parindent{0pt}
\usepackage{graphicx}
\usepackage{booktabs}
\usepackage{subcaption}
\usepackage{amsmath}
\usepackage{amsfonts}
\usepackage{amssymb}
\usepackage{lscape}
\usepackage{psfrag}
\usepackage{hyperref}
\hypersetup{
  colorlinks = false,
  urlcolor   = blue,
  linkcolor  = blue,
  citecolor  = red
}
\usepackage{verbatim}
\usepackage{textcomp}
\usepackage{multirow}
\usepackage{rotating}
\usepackage{adjustbox}
\usepackage{tikz}
\usepackage[english]{babel}
\usepackage{appendix}
\usepackage{parskip}
\usepackage{placeins}
\usepackage[tableposition=top]{caption}
\sloppy
\widowpenalty=10000
\clubpenalty=10000
\edef\today{%\number\day\
\ifcase\month\or
January\or February\or March\or April\or May\or June\or July\or
August\or September\or October\or November\or December\fi\ \number\year}
\title{\vspace*{40.0mm}
  \bf\sf Assignment
         \vspace*{20.0mm} \\
  \vspace*{40.0mm}}
\author{\sf Van den Broeck Sebastiaan (r0902562)}
\date{\sf 13/03/2023}

\begin{document}

\begin{figure}
  \parbox[t]{125mm}{
    \vspace*{6mm}
    \scriptsize\sf           FACULTY OF SCIENCE \\
    \scriptsize\sf           DEPARTMENT OF MATHEMATICS \\
    \scriptsize\sf\bfseries  MASTER OF STATISTICS AND DATA SCIENCE \\
    \scriptsize\sf\bfseries  STRUCTURAL EQUATION MODELING \\}
  \parbox[t]{40mm}{
    \begin{flushright}
      \includegraphics[height=15mm]{logo.eps.pdf}
    \end{flushright}}
\end{figure}

\maketitle
\thispagestyle{empty}
\raggedbottom

\cleardoublepage
\setcounter{page}{1}
\setcounter{tocdepth}{3}

\section{Introduction}

Harm avoidance and self-directedness have been linked to depression.
A behaviour can be classified under harm avoidance if it is done to avoid novelty and punishment.
Self-directedness, on the other hand, is a form of self-determination and ability to regulate behaviour to suit goals and values.
It has been proposed that harm avoidance and self-directedness are indirectly linked to depression through social functioning (\cite{tse2011}).
In this work I will test this hypothesis on a new dataset, which will be discussed next.
The structural equation model to test the hypothesis will be talked about.
Lastly, the results and implications thereof will be considered.

\section{Data}

The data treated in the report is the Midlife in the United States (MIDUS) series.
Currently, there are three waves in the study, which were collected via phone interviews, surveys and by bringing participants into clinical settings to facility collecting biological data.
All three waves cover the contiguous United States in its entirety.
The first wave was collected in 1995 and 1996, while the second wave was collected in 2004 and 2005.
The most recent wave was collected in 2013 and 2014.
The second and third wave have been combined to create a bigger dataset.
It was not possible to incorporate the first dataset, since a lot of variables changed between the first and second and third waves (\cite{radler2014}).

An important reason for choosing this dataset is that it contains a lot of documentation for which variables form certain latent constructs such as depression or social anxiety.
Since I am not familiar with the field of psychology this would save me a lot of time.
Depression is the most important latent variable in this work.
It has been measures through seven questions during which the respondent reflects over the last two weeks.
For example, the questions include losing interest, becoming tired, having trouble falling asleep or thinking about death.
Each variable which measures this latent construct has been coded such that a 1 reflects a yes answer.
As could be expected, a 0 then means a respondent has answered no.

\begin{table}[h!]
\scalebox{0.9}{
\begin{tabular}{|l|l|l|}
\hline
\textbf{Construct}          & \textbf{Code} & \textbf{Question}                                                                                                                                              \\ \hline
\multirow{7}{*}{Depression} & C1PA63        & During those two weeks, did you lose interest in most things?                                                                                                  \\ \cline{2-3} 
                            & C1P164        & \begin{tabular}[c]{@{}l@{}}Thinking about these same two weeks, did you feel more tired\\ out or low on energy?\end{tabular}                                   \\ \cline{2-3} 
                            & C1PA65        & During those same two weeks, did you lose appetite?                                                                                                            \\ \cline{2-3} 
                            & C1PA66        & \begin{tabular}[c]{@{}l@{}}Did you have more trouble falling asleep than you usually do\\ during those two weeks?\end{tabular}                                 \\ \cline{2-3} 
                            & C1PA67        & \begin{tabular}[c]{@{}l@{}}During that same two week period, did you have a lot more\\ trouble concentrating than usual?\end{tabular}                          \\ \cline{2-3} 
                            & C1PA68        & \begin{tabular}[c]{@{}l@{}}People sometimes feel down on themselves, no good, or worthless.\\ During that two-week period, did you feel this way?\end{tabular} \\ \cline{2-3} 
                            & C1PA69        & \begin{tabular}[c]{@{}l@{}}Did you think a lot about death - either your own, someone else's\\ or death in general - during those two weeks?\end{tabular}      \\ \hline
\end{tabular}
}
\end{table}

\begin{table}[h]
\scalebox{0.9}{
\begin{tabular}{|l|l|ll|}
\hline
\multirow{2}{*}{\textbf{Construct}} & \multirow{2}{*}{\textbf{Code}} & \multicolumn{2}{l|}{\textbf{Count}} \\ \cline{3-4} 
                                    &                                & \multicolumn{1}{l|}{0}      & 1     \\ \hline
\multirow{7}{*}{Depression}         & C1PA63                         & \multicolumn{1}{l|}{156}    & 633   \\ \cline{2-4} 
                                    & C1P164                         & \multicolumn{1}{l|}{61}     & 726   \\ \cline{2-4} 
                                    & C1PA65                         & \multicolumn{1}{l|}{338}    & 445   \\ \cline{2-4} 
                                    & C1PA66                         & \multicolumn{1}{l|}{223}    & 565   \\ \cline{2-4} 
                                    & C1PA67                         & \multicolumn{1}{l|}{111}    & 675   \\ \cline{2-4} 
                                    & C1PA68                         & \multicolumn{1}{l|}{283}    & 507   \\ \cline{2-4} 
                                    & C1PA69                         & \multicolumn{1}{l|}{304}    & 485   \\ \hline
\end{tabular}
}
\end{table}

Another important aspect in this report is harm avoidance. Since it cannot be measured directly, four questions were asked to get an idea about this variable.
First, interviewees were asked whether they would enjoy experiencing an earthquake or learning to walk the tightrope.
These two variables were reverse recoded such that a 4 reflects not agreeing with the statement at all (harm avoidance), while a 1 indicates fully agreeing (no avoidance).
Second, interviewees were presented with two scenario's twice.
For each question, one scenario corresponds to a harmful situation, while the other scenario's is harmless.
Again, there was a recoding such that a higher score on these two variables indicates avoiding harm.

\begin{table}[h!]
\scalebox{0.9}{
\begin{tabular}{|l|l|l|}
\hline
\textbf{Construct}              & \textbf{Code} & \textbf{Question}                                                                                                                                                                                                                                                                                                              \\ \hline
\multirow{4}{*}{Harm avoidance} & C1SE7D        & It might be fun and exciting to be in an earthquake.                                                                                                                                                                                                                                                                           \\ \cline{2-3} 
                                & C1SE7V        & It might be fun learning to walk a tightrope.                                                                                                                                                                                                                                                                                  \\ \cline{2-3} 
                                & C1SE8         & \begin{tabular}[c]{@{}l@{}}Of these two situations, I would dislike more: Situation 1: \\ Riding a long stretch of rapids in a canoe; Situation 2:\\ Waiting for someone who's late.\end{tabular}                                                                                                                              \\ \cline{2-3} 
                                & C1SE9         & \begin{tabular}[c]{@{}l@{}}Of these two situations, I would dislike more: Situation 1:\\ Being at the circus when two lions suddenly get loose\\ down in the ring; Situation 2: Bringing my whole family\\ to the circus and then not being able to get in because a\\ clerk sold me tickets for the wrong night.\end{tabular} \\ \hline
\end{tabular}
}
\end{table}

\begin{table}[h!]
\scalebox{0.9}{
\begin{tabular}{|l|l|llll|}
\hline
\multirow{2}{*}{\textbf{Construct}} & \multirow{2}{*}{\textbf{Code}} & \multicolumn{4}{l|}{\textbf{Count}}                                                                              \\ \cline{3-6} 
                                    &                                & \multicolumn{1}{l|}{\textbf{1 (harm)}} & \multicolumn{1}{l|}{\textbf{2}} & \multicolumn{1}{l|}{\textbf{3}} & \textbf{4 (no harm)} \\ \hline
\multirow{2}{*}{Harm avoidance}     & C1SE7D                         & \multicolumn{1}{l|}{274}               & \multicolumn{1}{l|}{875}        & \multicolumn{1}{l|}{838}        & 4889                 \\ \cline{2-6} 
                                    & C1SE7V                         & \multicolumn{1}{l|}{367}               & \multicolumn{1}{l|}{1222}       & \multicolumn{1}{l|}{1163}       & 4238                 \\ \hline
\end{tabular}
}
\end{table}

\begin{table}[h!]
\scalebox{0.9}{
\begin{tabular}{|l|l|ll|}
\hline
\multirow{2}{*}{\textbf{Construct}} & \multirow{2}{*}{\textbf{Code}} & \multicolumn{2}{l|}{\textbf{Count}}          \\ \cline{3-4} 
                                    &                                & \multicolumn{1}{l|}{\textbf{0 (harm)}} & \textbf{1 (no harm)} \\ \hline
\multirow{2}{*}{Harm avoidance}     & C1SE8                          & \multicolumn{1}{l|}{3803}                 & 3089       \\ \cline{2-4} 
                                    & C1SE7V                         & \multicolumn{1}{l|}{2994}                 & 3898       \\ \hline
\end{tabular}
}
\end{table}

We should not forget about self-directedness, which has been measured through three variables.
Making plans for the future, knowing what to want out of life and setting goals are important for this dimension.
Again, the variables were reverse coded such that a higher score reflects agreeing more with the statement.
The data indicates that most participants agree somewhat or fully what the three statements.

\begin{table}[h!]
\scalebox{0.9}{
\begin{tabular}{|l|l|l|}
\hline
\textbf{Construct}                 & \textbf{Code} & \textbf{Question}                                   \\ \hline
\multirow{3}{*}{Self-directedness} & C1SE14O       & I like to make plans for the future.                \\ \cline{2-3} 
                                   & C1SE14R       & I know what I want out of life.                     \\ \cline{2-3} 
                                   & C1SE14P       & I find it helpful to set goals for the near future. \\ \hline
\end{tabular}
}
\end{table}

\begin{table}[h!]
\scalebox{0.9}{
\begin{tabular}{|l|l|llll|}
\hline
\multirow{2}{*}{\textbf{Construct}} & \multirow{2}{*}{\textbf{Code}} & \multicolumn{4}{l|}{\textbf{Count}}                                                                              \\ \cline{3-6} 
                                    &                                & \multicolumn{1}{l|}{\textbf{1}} & \multicolumn{1}{l|}{\textbf{2}} & \multicolumn{1}{l|}{\textbf{3}} & \textbf{4} \\ \hline
\multirow{3}{*}{Self-directedness}  & C1SE14O                        & \multicolumn{1}{l|}{247}        & \multicolumn{1}{l|}{1303}       & \multicolumn{1}{l|}{2754}       & 2590       \\ \cline{2-6} 
                                    & C1SE14R                        & \multicolumn{1}{l|}{251}        & \multicolumn{1}{l|}{1089}       & \multicolumn{1}{l|}{2929}       & 2604       \\ \cline{2-6} 
                                    & C1SE14P                        & \multicolumn{1}{l|}{318}        & \multicolumn{1}{l|}{1320}       & \multicolumn{1}{l|}{3071}       & 2184       \\ \hline
\end{tabular}
}
\end{table}

Lastly, the latent variable social functioning has been used in the analysis.
Seven questions related to this dimension were asked.
The variables C1SE1BB, C1SE1D, C1SE1I and C1SE1V were reverse coded such that a higher score indicates a higher degree of social functioning.

\begin{table}[h!]
\scalebox{0.9}{
\begin{tabular}{|l|l|l|}
\hline
\textbf{Construct}                  & \textbf{Code} & \textbf{Question}                                                                                                                                    \\ \hline
\multirow{7}{*}{Social functioning} & C1SE1BB       & \begin{tabular}[c]{@{}l@{}}People would describe me as a giving person, willing to share my time \\ with others.\end{tabular}                        \\ \cline{2-3} 
                                    & C1SE1D        & Most people see me as loving and affectionate.                                                                                                       \\ \cline{2-3} 
                                    & C1SE1HH       & I have not experienced many warm and trusting relationships with others.                                                                             \\ \cline{2-3} 
                                    & C1SE1J        & Maintaining close relationships has been difficult and frustrating for me.                                                                           \\ \cline{2-3} 
                                    & C1SE1I        & \begin{tabular}[c]{@{}l@{}}I think it is important to have new experiences that challenge how you think\\ about yourself and the world.\end{tabular} \\ \cline{2-3} 
                                    & C1SE1P        & \begin{tabular}[c]{@{}l@{}}I often feel lonely because I have few close friends with whom to share my\\ concerns.\end{tabular}                       \\ \cline{2-3} 
                                    & C1SE1V        & I enjoy personal and mutual conversations with family members and friends.                                                                           \\ \hline
\end{tabular}
}
\end{table}

\begin{table}[h!]
\begin{tabular}{|l|l|lllllll|}
\hline
\multirow{2}{*}{\textbf{Construct}} & \multirow{2}{*}{\textbf{Code}} & \multicolumn{7}{l|}{\textbf{Count}}                                                                                                                                                                                    \\ \cline{3-9} 
                                    &                                & \multicolumn{1}{l|}{\textbf{1}} & \multicolumn{1}{l|}{\textbf{2}} & \multicolumn{1}{l|}{\textbf{3}} & \multicolumn{1}{l|}{\textbf{4}} & \multicolumn{1}{l|}{\textbf{5}} & \multicolumn{1}{l|}{\textbf{6}} & \textbf{7} \\ \hline
\multirow{7}{*}{Social functioning} & C1SE1BB                        & \multicolumn{1}{l|}{28}         & \multicolumn{1}{l|}{76}         & \multicolumn{1}{l|}{127}        & \multicolumn{1}{l|}{472}        & \multicolumn{1}{l|}{799}        & \multicolumn{1}{l|}{2382}       & 3055       \\ \cline{2-9} 
                                    & C1SE1D                         & \multicolumn{1}{l|}{43}         & \multicolumn{1}{l|}{126}        & \multicolumn{1}{l|}{219}        & \multicolumn{1}{l|}{807}        & \multicolumn{1}{l|}{815}        & \multicolumn{1}{l|}{2598}       & 230        \\ \cline{2-9} 
                                    & C1SE1HH                        & \multicolumn{1}{l|}{291}        & \multicolumn{1}{l|}{501}        & \multicolumn{1}{l|}{575}        & \multicolumn{1}{l|}{441}        & \multicolumn{1}{l|}{497}        & \multicolumn{1}{l|}{1382}       & 3242       \\ \cline{2-9} 
                                    & C1SE1J                         & \multicolumn{1}{l|}{236}        & \multicolumn{1}{l|}{575}        & \multicolumn{1}{l|}{818}        & \multicolumn{1}{l|}{711}        & \multicolumn{1}{l|}{512}        & \multicolumn{1}{l|}{1524}       & 2550       \\ \cline{2-9} 
                                    & C1SE1I                         & \multicolumn{1}{l|}{113}        & \multicolumn{1}{l|}{136}        & \multicolumn{1}{l|}{142}        & \multicolumn{1}{l|}{709}        & \multicolumn{1}{l|}{1154}       & \multicolumn{1}{l|}{2190}       & 2483       \\ \cline{2-9} 
                                    & C1SE1P                         & \multicolumn{1}{l|}{242}        & \multicolumn{1}{l|}{447}        & \multicolumn{1}{l|}{744}        & \multicolumn{1}{l|}{655}        & \multicolumn{1}{l|}{479}        & \multicolumn{1}{l|}{1321}       & 3036       \\ \cline{2-9} 
                                    & C1SE1V                         & \multicolumn{1}{l|}{63}         & \multicolumn{1}{l|}{64}         & \multicolumn{1}{l|}{105}        & \multicolumn{1}{l|}{211}        & \multicolumn{1}{l|}{589}        & \multicolumn{1}{l|}{1879}       & 4024       \\ \hline
\end{tabular}
\end{table}


\FloatBarrier
\pagebreak
\section{The base model}

\begin{}

\FloatBarrier
\pagebreak
\section{Expanding the base model}

\FloatBarrier
\pagebreak
\section{Conclusion}

\end{document}
