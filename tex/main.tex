%!TEX program = xelatex

\documentclass[11pt]{article}
\usepackage{geometry}
\usepackage{tcolorbox}
\usepackage{hyperref}
\usepackage{microtype}
\usepackage{rotating}
\usepackage[backend=biber,sorting=none,style=apa]{biblatex}
\addbibresource{library.bib}
\geometry{
    a4paper,
    total={170mm,257mm},
    left=20mm,
    top=20mm,
}
\setlength{\parskip}{5pt}
\setlength\parindent{0pt}
\usepackage{graphicx}
\usepackage{booktabs}
\usepackage{subcaption}
\usepackage{amsmath}
\usepackage{amsfonts}
\usepackage{amssymb}
\usepackage{lscape}
\usepackage{psfrag}
\usepackage{hyperref}
\hypersetup{
  colorlinks = false,
  urlcolor   = blue,
  linkcolor  = blue,
  citecolor  = red
}
\usepackage{verbatim}
\usepackage{textcomp}
\usepackage{multirow}
\usepackage{rotating}
\usepackage{adjustbox}
\usepackage{tikz}
\usepackage[english]{babel}
\usepackage{appendix}
\usepackage{parskip}
\usepackage{placeins}
\usepackage[tableposition=top]{caption}
\captionsetup{skip=0pt}

\usepackage{amsmath}
\usepackage{fontspec}
\usepackage[charter]{newtxmath}

\setcounter{MaxMatrixCols}{30}

\setmainfont{XCharter}
\newfontfamily{\XCharterLF}{XCharter}[NFSSFamily=xcharterlf]

\DeclareSymbolFont{operators}{TU}{xcharterlf}{m}{n}

\usepackage{listings}
\lstset{
  language=R,
  basicstyle=\footnotesize,
  numbers=left,
  numberstyle=\tiny,
  numbersep=5pt,
  showstringspaces=false,
  frame=single
} 

\sloppy
\widowpenalty=10000
\clubpenalty=10000
\edef\today{%\number\day\
\ifcase\month\or
January\or February\or March\or April\or May\or June\or July\or
August\or September\or October\or November\or December\fi\ \number\year}
\title{\vspace*{40.0mm}
  \bf\sf Assignment
         \vspace*{20.0mm} \\
  \vspace*{40.0mm}}
\author{\sf Sebastiaan Van den Broeck (r0902562)}
\date{\sf 28/05/2023}

\begin{document}

\begin{figure}
  \parbox[t]{125mm}{
    \vspace*{6mm}
    \scriptsize\sf           FACULTY OF SCIENCE \\
    \scriptsize\sf           DEPARTMENT OF MATHEMATICS \\
    \scriptsize\sf\bfseries  MASTER OF STATISTICS AND DATA SCIENCE \\
    \scriptsize\sf\bfseries  STRUCTURAL EQUATIONS \\}
  \parbox[t]{40mm}{
    \begin{flushright}
      \includegraphics[height=15mm]{logo.eps.pdf}
    \end{flushright}}
\end{figure}

\maketitle
\thispagestyle{empty}
\raggedbottom

\cleardoublepage
\setcounter{page}{1}
\setcounter{tocdepth}{3}

\tableofcontents
\listoffigures
\listoftables

\pagebreak\section{Introduction}

It has been noted that almost one in six individuals in the United States will
experience a depressive disorder. Consequently, considerable personal, social
and economic loss can be attributed to this type of illness. Although there is
clearly a very large societal impact of depression, little is known about its
relationship with personality and social functioning. In this work, I will test
a hypothesis related to depression that has been proposed by \textcite{tse2011}.
Specifically, they proposed that harm avoidance and self-directedness are
indirectly linked to depression through social functioning. Moreover, there
should also be a direct effect of self-directedness on depression. On the one
hand, a behaviour can be classified under harm avoidance if it is done to avoid
novelty and punishment. Self-directedness, on the other hand, is a form of
self-determination and ability to regulate behaviour to suit goals and values.
The authors have tested this hypothesis on a sample of university students,
which limits the interpretability of their findings. By testing their hypothesis
on a larger and more representative sample, I hope to contribute to the
literature on depression. First, some preliminaries will be discussed: the data
and measurement invariance. Afterwards, a structural equation model has been
used to test the hypothesis and will be discussed as well. Lastly, the results
and implications thereof will be considered.

\section{Data}

The data treated in this report is the Midlife in the United States (MIDUS)
series. It is a national study of health and well-being, created by a team of
multi-disciplinary researchers. Currently, there are three waves in the study,
which were collected via phone interviews, surveys and by bringing participants
into clinical settings to facilitate collecting biological data. All three waves
cover the contiguous United States in its entirety. The first wave was collected
in 1995 and 1996, while the second wave was collected in 2004 and 2005. The most
recent and third wave was collected in 2013 and 2014. In this analysis, the
second and third waves have been combined to create a bigger dataset. It was not
possible to incorporate the first wave, since a lot of variables changed between
the first and second and third waves (\cite{radler2014}). In this section, I
will discuss the variables that have been used in the analysis.

An important reason for choosing this dataset is that it contains a lot of
documentation for which variables form certain latent constructs such as
depression or social anxiety. I have to admit that I don't have a lot of
experience with the field of psychology, so this documentation was very helpful
and allows me to test a hypothesis that is better grounded in theory. First,
depression is the most important latent variable in this work. It has been
measured through seven questions during which the respondent reflects over their
last two weeks. For example, the questions include losing interest, becoming
tired, having trouble falling asleep or thinking about death. The responses have
been recoded such that a higher score equates a higher level of depression.
Specifically, each variable which measures this latent construct has been coded
such that a 1 reflects a yes answer. As could be expected, a 0 then means a
respondent has answered no.

\begin{table}[h!]
\captionsetup{singlelinecheck=off}
\caption{Depression indicators}
\scalebox{0.9}{
\begin{tabular}{|l|l|l|}
\hline
\textbf{Construct}          & \textbf{Code} & \textbf{Question}                                                                                                                                              \\ \hline
\multirow{7}{*}{Depression} & PA63        & During those two weeks, did you lose interest in most things?                                                                                                  \\ \cline{2-3} 
                            & PA64        & \begin{tabular}[c]{@{}l@{}}Thinking about these same two weeks, did you feel more tired\\ out or low on energy?\end{tabular}                                   \\ \cline{2-3} 
                            & PA65        & During those same two weeks, did you lose appetite?                                                                                                            \\ \cline{2-3} 
                            & PA66        & \begin{tabular}[c]{@{}l@{}}Did you have more trouble falling asleep than you usually do\\ during those two weeks?\end{tabular}                                 \\ \cline{2-3} 
                            & PA67        & \begin{tabular}[c]{@{}l@{}}During that same two week period, did you have a lot more\\ trouble concentrating than usual?\end{tabular}                          \\ \cline{2-3} 
                            & PA68        & \begin{tabular}[c]{@{}l@{}}People sometimes feel down on themselves, no good, or worthless.\\ During that two-week period, did you feel this way?\end{tabular} \\ \cline{2-3} 
                            & PA69        & \begin{tabular}[c]{@{}l@{}}Did you think a lot about death - either your own, someone else's\\ or death in general - during those two weeks?\end{tabular}      \\ \hline
\end{tabular}
}
\end{table}

\begin{table}[h!]
\captionsetup{singlelinecheck=off}
\caption{Depression distribution}
\scalebox{0.9}{
\begin{tabular}{|l|l|ll|}
\hline
\multirow{2}{*}{\textbf{Construct}} & \multirow{2}{*}{\textbf{Code}} & \multicolumn{2}{l|}{\textbf{Count}} \\ \cline{3-4} 
                                    &                                & \multicolumn{1}{l|}{0}      & 1     \\ \hline
\multirow{7}{*}{Depression}         & PA63                         & \multicolumn{1}{l|}{126}    & 479   \\ \cline{2-4} 
                                    & PA64                         & \multicolumn{1}{l|}{51}     & 554   \\ \cline{2-4} 
                                    & PA65                         & \multicolumn{1}{l|}{263}    & 342   \\ \cline{2-4} 
                                    & PA66                         & \multicolumn{1}{l|}{172}    & 433   \\ \cline{2-4} 
                                    & PA67                         & \multicolumn{1}{l|}{88}     & 517   \\ \cline{2-4} 
                                    & PA68                         & \multicolumn{1}{l|}{222}    & 383   \\ \cline{2-4} 
                                    & PA69                         & \multicolumn{1}{l|}{229}    & 376   \\ \hline
\end{tabular}
}
\end{table}

\FloatBarrier

Second, another important aspect in this report is harm avoidance. It has been
described as an inheritable tendency for inhibiting behaviours to avoid novelty
and punishment (\cite{tse2011}). Since it cannot be measured directly, four
questions were asked to get an idea about this construct. First, interviewees
were asked whether they would enjoy experiencing an earthquake or learning to
walk the tightrope. These two variables were reverse recoded such that a 4
reflects not agreeing with the statement at all (harm avoidance), while a 1
indicates fully agreeing (no avoidance). Second, interviewees were presented
with two scenario's twice. For each question, one scenario corresponds to a
harmful situation, while the other scenario's is harmless. Again, there was a
recoding such that a higher score on these two variables indicates avoiding
harm.

\begin{table}[h]
\captionsetup{singlelinecheck=off}
\caption{Harm avoidance indicators}
\scalebox{0.9}{
\begin{tabular}{|l|l|l|}
\hline
\textbf{Construct}              & \textbf{Code} & \textbf{Question}                                                                                                                                                                                                                                                                                                              \\ \hline
\multirow{4}{*}{Harm avoidance} & SE7D        & It might be fun and exciting to be in an earthquake.                                                                                                                                                                                                                                                                           \\ \cline{2-3} 
                                & SE7V        & It might be fun learning to walk a tightrope.                                                                                                                                                                                                                                                                                  \\ \cline{2-3} 
                                & SE8         & \begin{tabular}[c]{@{}l@{}}Of these two situations, I would dislike more: Situation 1: \\ Riding a long stretch of rapids in a canoe; Situation 2:\\ Waiting for someone who's late.\end{tabular}                                                                                                                              \\ \cline{2-3} 
                                & SE9         & \begin{tabular}[c]{@{}l@{}}Of these two situations, I would dislike more: Situation 1:\\ Being at the circus when two lions suddenly get loose\\ down in the ring; Situation 2: Bringing my whole family\\ to the circus and then not being able to get in because a\\ clerk sold me tickets for the wrong night.\end{tabular} \\ \hline
\end{tabular}
}
\end{table}

\begin{table}[h]
\begin{minipage}[b]{0.55\linewidth}
\captionsetup{singlelinecheck=off}
\caption{Harm avoidance distribution}
\scalebox{0.8}{
\begin{tabular}{|l|l|llll|}
\hline
\multirow{2}{*}{\textbf{Construct}} & \multirow{2}{*}{\textbf{Code}} & \multicolumn{4}{l|}{\textbf{Count}}                                                                              \\ \cline{3-6} 
                                    &                                & \multicolumn{1}{l|}{\textbf{1 (harm)}} & \multicolumn{1}{l|}{\textbf{2}} & \multicolumn{1}{l|}{\textbf{3}} & \textbf{4 (no harm)} \\ \hline
\multirow{2}{*}{Harm avoidance}     & SE7D                           & \multicolumn{1}{l|}{33}                  & \multicolumn{1}{l|}{92}         & \multicolumn{1}{l|}{91}         & 389                 \\ \cline{2-6} 
                                    & SE7V                           & \multicolumn{1}{l|}{25}                  & \multicolumn{1}{l|}{79}         & \multicolumn{1}{l|}{69}         & 432                 \\ \hline
\end{tabular}
}
\end{minipage}
\begin{minipage}[b]{0.4\linewidth}
\scalebox{0.8}{
\begin{tabular}{|l|l|ll|}
\hline
\multirow{2}{*}{\textbf{Construct}} & \multirow{2}{*}{\textbf{Code}} & \multicolumn{2}{l|}{\textbf{Count}}          \\ \cline{3-4} 
                                    &                                & \multicolumn{1}{l|}{\textbf{0 (harm)}} & \textbf{1 (no harm)} \\ \hline
\multirow{2}{*}{Harm avoidance}     & SE8                            & \multicolumn{1}{l|}{335}                  & 270       \\ \cline{2-4} 
                                    & SE7V                           & \multicolumn{1}{l|}{276}                  & 329       \\ \hline
\end{tabular}
}
\end{minipage}
\end{table}

Third, we should not forget about self-directedness, which has been measured
through three variables. It evaluates the amount of self-determination and
ability a respondent has in order to regulate behaviour to achieve goals and
values (\cite{tse2011}). Making plans for the future, knowing what to want out
of life and setting goals are important for this dimension. Again, the variables
were reverse coded such that a higher score reflects agreeing more with the
statement. The data indicates that most participants agree somewhat or fully the
three statements.

\begin{table}[h!]
\captionsetup{singlelinecheck=off}
\caption{Self-directedness indicators}
\scalebox{0.8}{
\begin{tabular}{|l|l|l|}
\hline
\textbf{Construct}                 & \textbf{Code} & \textbf{Question}                                   \\ \hline
\multirow{3}{*}{Self-directedness} & SE14O       & I like to make plans for the future.                \\ \cline{2-3} 
                                   & SE14R       & I know what I want out of life.                     \\ \cline{2-3} 
                                   & SE14P       & I find it helpful to set goals for the near future. \\ \hline
\end{tabular}
}
\end{table}

\begin{table}[h!]
\captionsetup{singlelinecheck=off}
\caption{Self-directedness distribution}
\scalebox{0.8}{
\begin{tabular}{|l|l|llll|}
\hline
\multirow{2}{*}{\textbf{Construct}} & \multirow{2}{*}{\textbf{Code}} & \multicolumn{4}{l|}{\textbf{Count}}                                                                              \\ \cline{3-6} 
                                    &                                & \multicolumn{1}{l|}{\textbf{1}} & \multicolumn{1}{l|}{\textbf{2}} & \multicolumn{1}{l|}{\textbf{3}} & \textbf{4} \\ \hline
\multirow{3}{*}{Self-directedness}  & SE14O                        & \multicolumn{1}{l|}{48}        & \multicolumn{1}{l|}{140}       & \multicolumn{1}{l|}{241}       & 176       \\ \cline{2-6} 
                                    & SE14R                        & \multicolumn{1}{l|}{67}        & \multicolumn{1}{l|}{144}       & \multicolumn{1}{l|}{235}       & 159       \\ \cline{2-6} 
                                    & SE14P                        & \multicolumn{1}{l|}{47}        & \multicolumn{1}{l|}{149}       & \multicolumn{1}{l|}{228}       & 181       \\ \hline
\end{tabular}
}
\end{table}

Fourth, the latent variable social functioning has been used in the analysis.
Seven questions related to this dimension were asked. The variables SE1BB, SE1D,
SE1I and SE1V were reverse coded such that a higher score indicates a higher
degree of social functioning.  The dataset counts 601 observations after
deleting rows with missing values. A missingness at random principle is
therefore assumed.

\begin{table}[h!]
\captionsetup{singlelinecheck=off}
\caption{Social functioning indicators}
\scalebox{0.8}{
\begin{tabular}{|l|l|l|}
\hline
\textbf{Construct}                  & \textbf{Code} & \textbf{Question}                                                                                                                                    \\ \hline
\multirow{7}{*}{Social functioning} & SE1BB         & \begin{tabular}[c]{@{}l@{}}People would describe me as a giving person, willing to share my time \\ with others.\end{tabular}                        \\ \cline{2-3} 
                                    & SE1D          & Most people see me as loving and affectionate.                                                                                                       \\ \cline{2-3} 
                                    & SE1HH         & I have not experienced many warm and trusting relationships with others.                                                                             \\ \cline{2-3} 
                                    & SE1J          & Maintaining close relationships has been difficult and frustrating for me.                                                                           \\ \cline{2-3} 
                                    & SE1P          & \begin{tabular}[c]{@{}l@{}}I often feel lonely because I have few close friends with whom to share my\\ concerns.\end{tabular}                       \\ \cline{2-3} 
                                    & SE1V          & I enjoy personal and mutual conversations with family members and friends.                                                                           \\ \hline
\end{tabular}
}
\end{table}

\begin{table}[h!]
\captionsetup{singlelinecheck=off}
\caption{Social functioning distribution}
\scalebox{0.8}{
\begin{tabular}{|l|l|lllllll|}
\hline
\multirow{2}{*}{\textbf{Construct}} & \multirow{2}{*}{\textbf{Code}} & \multicolumn{7}{l|}{\textbf{Count}}                                                                                                                                                                                    \\ \cline{3-9} 
                                    &                                & \multicolumn{1}{l|}{\textbf{1}} & \multicolumn{1}{l|}{\textbf{2}} & \multicolumn{1}{l|}{\textbf{3}} & \multicolumn{1}{l|}{\textbf{4}} & \multicolumn{1}{l|}{\textbf{5}} & \multicolumn{1}{l|}{\textbf{6}} & \textbf{7} \\ \hline
\multirow{7}{*}{Social functioning} & SE1BB                          & \multicolumn{1}{l|}{4}          & \multicolumn{1}{l|}{7}          & \multicolumn{1}{l|}{16}         & \multicolumn{1}{l|}{36}         & \multicolumn{1}{l|}{70}         & \multicolumn{1}{l|}{207}       & 265       \\ \cline{2-9} 
                                    & SE1D                           & \multicolumn{1}{l|}{5}          & \multicolumn{1}{l|}{14}         & \multicolumn{1}{l|}{23}         & \multicolumn{1}{l|}{68}         & \multicolumn{1}{l|}{74}         & \multicolumn{1}{l|}{225}       & 196        \\ \cline{2-9} 
                                    & SE1HH                          & \multicolumn{1}{l|}{75}         & \multicolumn{1}{l|}{71}         & \multicolumn{1}{l|}{69}         & \multicolumn{1}{l|}{34}         & \multicolumn{1}{l|}{51}         & \multicolumn{1}{l|}{113}       & 192       \\ \cline{2-9} 
                                    & SE1J                           & \multicolumn{1}{l|}{67}         & \multicolumn{1}{l|}{88}         & \multicolumn{1}{l|}{106}        & \multicolumn{1}{l|}{57}         & \multicolumn{1}{l|}{52}         & \multicolumn{1}{l|}{94}        & 141       \\ \cline{2-9} 
                                    & SE1I                           & \multicolumn{1}{l|}{7}          & \multicolumn{1}{l|}{14}         & \multicolumn{1}{l|}{14}         & \multicolumn{1}{l|}{57}         & \multicolumn{1}{l|}{124}        & \multicolumn{1}{l|}{175}       & 215       \\ \cline{2-9} 
                                    & SE1P                           & \multicolumn{1}{l|}{71}         & \multicolumn{1}{l|}{82}         & \multicolumn{1}{l|}{80}         & \multicolumn{1}{l|}{51}         & \multicolumn{1}{l|}{50}         & \multicolumn{1}{l|}{103}       & 168       \\ \cline{2-9} 
                                    & SE1V                           & \multicolumn{1}{l|}{14}         & \multicolumn{1}{l|}{14}         & \multicolumn{1}{l|}{17}         & \multicolumn{1}{l|}{26}         & \multicolumn{1}{l|}{82}         & \multicolumn{1}{l|}{159}       & 293       \\ \hline
\end{tabular}
}
\end{table}

Lastly, the relationships between the variables themselves will be shortly
considered. The correlation matrix is shown in Figure \ref{fig:corr}. The
polychoric correlations play a pivotal role in this work and will be discussed
in the next section. For illustration purposes, they have been shown in Figure
\ref{fig:poly_corr}. They are a little bit stronger than the traditional
correlations.
%
First, we may notice some triangles near the diagonal. This is a good sign that
the indicators of a construct are correlated with each other. Convergent
validity is important in the context of structural equation modeling and means
that indicators that load on the same factor should be strongly related to each
other (\cite{brown2015}). In the case of the present study, the indicators of
the depression latent variable (first triangle) do not appear to be strongly
correlated with each other. This may present a problem for the model, which will
be further discussed later. Second, it is clear that the variables of
self-directedness and social functioning are negatively correlated. This is a
good sign as well, since the structural equation model will be able to
capitalize on this relationship. Moreover, it appears that some of the
depression and social functioning indicators are negatively correlated as well.

\begin{figure}[h!]
\centering
  \begin{minipage}{0.49\linewidth}
  \includegraphics[width=\linewidth]{../visualizations/corr.png}
  \caption{Correlation plot}
  \label{fig:corr}
  \end{minipage}
  \hfill
  \begin{minipage}{0.49\linewidth}
  \includegraphics[width=\linewidth]{../visualizations/corr_poly.png}
  \caption{Polychoric correlation plot}
  \label{fig:poly_corr}
  \end{minipage}
\end{figure}

\FloatBarrier\pagebreak\section{Structural equation model}

Next, the full structural equation model will be discussed. In opposition to the
previous section the structural model will be added to the mix. Based on the
work of \textcite{tse2011}, it has been concluded that depression can be
explained through the constructs harm avoidance, self directedness and social
functioning. In other words, there are four latent variables which are related
to each other through a structural model. Harm avoidance and self-directedness
have an effect on social functioning. Social functioning, then, has an effect on
depression. Moreover, it was estimated that there is also a direct effect of
self-directedness on depression. Hence, there should be no direct effect of harm
avoidance on depression.

In the previous section it became clear that ordinal variables have been used in
the analysis. Model identification and estimation will therefore be discussed
first. The measurement model indicates how variables are related to their latent
constructs and will be discussed next. Afterwards, a closer look will be taken
at the structural model, since an important aspect in this work is the
relationships between latent variables. Lastly, the model fit will be evaluated
through fit measures and further inspected using modification indices.

\subsection{Model identification and estimation}

% Model identification
First and foremost, model identification and estimation will be discussed.
A structural equation model is said to be identified if every latent variable
has its scale identified and the models degrees of freedom is zero or greater.
To that end, the scale of the first indicator of every latent variable has been
fixed to one. The model contains 85 parameters that should be estimated.
Specifically, there are 16 ($=20 - 4$) loadings, 4 regression parameters, 60
thresholds (more on this later), 1 latent covariance and 4 latent variances.
The degrees of freedom of the model therefore equals 20*21/2 - 85 = 210 - 85
= 125. However, Lavaan reports that the degrees of freedom equals 165. Honestly,
I don't know why this is the case. I have tried to find an explanation in the
documentation of Lavaan, but I have not been able to find one.
% 20*23/2-85 = 230-85 = 145
% 20*25/2-85 = 250-85 = 165
Next, the following assumptions have been made on the equations shown
in \ref{eq:base_model}. The measurement errors $\delta$ are supposed to have an
expected value of 0. It is assumed that they have constant variance across
observations and are mutually uncorrelated. There should be a covariance of zero
between these errors and the latent variables.

\begin{equation}
    \footnotesize
    \label{eq:base_model}
    \begin{cases}
    \textrm{PA63}  & = \lambda_{11} \textrm{depression}         + \theta_{11} \\
    \textrm{PA64}  & = \lambda_{12} \textrm{depression}         + \theta_{12} \\
    \textrm{PA65}  & = \lambda_{13} \textrm{depression}         + \theta_{13} \\
    \textrm{PA66}  & = \lambda_{14} \textrm{depression}         + \theta_{14} \\
    \textrm{PA67}  & = \lambda_{15} \textrm{depression}         + \theta_{15} \\
    \textrm{PA68}  & = \lambda_{16} \textrm{depression}         + \theta_{16} \\
    \textrm{PA69}  & = \lambda_{17} \textrm{depression}         + \theta_{17} \\
    \textrm{SE7V}  & = \lambda_{21} \textrm{harm avoidance}     + \theta_{21} \\
    \textrm{SE7D}  & = \lambda_{22} \textrm{harm avoidance}     + \theta_{22} \\
    \textrm{SE8}   & = \lambda_{23} \textrm{harm avoidance}     + \theta_{23} \\
    \textrm{SE9}   & = \lambda_{24} \textrm{harm avoidance}     + \theta_{24} \\
    \textrm{SE14O} & = \lambda_{31} \textrm{self-directedness}  + \theta_{31} \\
    \textrm{SE14P} & = \lambda_{32} \textrm{self-directedness}  + \theta_{32} \\
    \textrm{SE14R} & = \lambda_{33} \textrm{self-directedness}  + \theta_{33} \\
    \textrm{SE1BB} & = \lambda_{41} \textrm{social functioning} + \theta_{41} \\
    \textrm{SE1D}  & = \lambda_{42} \textrm{social functioning} + \theta_{42} \\
    \textrm{SE1HH} & = \lambda_{43} \textrm{social functioning} + \theta_{43} \\
    \textrm{SE1J}  & = \lambda_{44} \textrm{social functioning} + \theta_{44} \\
    \textrm{SE1P}  & = \lambda_{45} \textrm{social functioning} + \theta_{45} \\
    \textrm{SE1V}  & = \lambda_{46} \textrm{social functioning} + \theta_{46} \\
    \textrm{social functioning} & = \beta_1 \textrm{harm avoidance} + \beta_2 \textrm{self-directedness} + \psi_1 \\
    \textrm{depression} & = \beta_3 \textrm{social functioning} + \beta_4 \textrm{self-directedness} + \psi_2
    \end{cases}
    \iff
    \begin{cases}
      \underset{n \times n}{\pmb{x}} = \underset{n \times p}{\pmb{\Lambda}} \underset{?}{\pmb{\xi}} + \underset{n \times n}{\pmb{\theta}} \\
      \underset{p \times p}{\pmb{y}} = \underset{p \times p}{\pmb{\beta}} \underset{?}{\pmb{\xi}} + \underset{p \times p}{\pmb{\psi}} \\
      
      \pmb{\theta} \sim \mathcal{N} (0, \Sigma_{\pmb{\theta}}) \\
      \pmb{\psi} \sim \mathcal{N} (0, \Sigma_{\pmb{\psi}}) \\
      
      n = 20 \hfill \text{(number of variables)}        \\
      p = 4  \hfill \text{(number of latent variables)} \\

    \end{cases}
\end{equation}


% TODO: add thing about software

A considerable problem arises when one considers the assumption of multivariate
normality on the residuals $\pmb{\theta}$. All observed variables are ordinal in
nature, meaning that they are not continuous and should not be treated as such.
Their means and (co)variances have no meaning, since they do not have origins or
units of measurement (\cite{joreskog1994}). The standard maximum likelihood
machinery used in SEM is therefore also not applicable. It would therefore be
questionable to take the usual approach in SEM of modelling the covariance
matrix, although it has been shown that MLE can be used in an ordinal context
under certain conditions. In this case, robust maximum likelihood or a least
squares approach (unweighted least squares, diagonally weighted least squares or
weighted least squares) can be used (\cite{yangwallentin2010}). The method of
diagonally weighted least squares for SEM has been specifically developed for
ordinal data and has been shown to yield better results when the sample size is
not small (\cite{li2016}). It has therefore been applied here. First, polychoric
correlations are estimated. Afterwards, the model parameters can be estimated. 

First, the polychoric correlations should be estimated. A solution can then be
obtained by assuming that a latent, normal variable $x^*$ is responsible for the
observed ordinal variables $x$. With $x=m$ I mean to say that $x$ belongs to a
category $m$. Generally, the mean and variance if $x^*$ are not identified,
since only ordinal information is available (\cite{simsek2012}). They are
therefore fixed to zero and one. Thresholds $\nu$ are used to link the latent
variable to its observed counterpart:

\begin{equation}
  x = m \:\: \text{if} \:\: \nu_m < x^* < \nu_{m+1} .
\end{equation}

Also, if one assumes $x^*$ is standard normally distributed and $\phi$ and
$\Phi$ denote the standard normal density and distribution functions:

\begin{equation}
  P[x=m] = P[\nu_m < x^* < \nu_{m+1}] = \int^{\nu_{m+1}}_{\nu_m} \phi(u)du = \Phi(\nu_{m+1}) - \Phi(\nu_m) .
\end{equation}

In other words, a certain response $m$ from the ordinal variable $x$ is
observed, if the response from its latent variable $x^*$ falls between two
thresholds. Hence, the thresholds are also parameters to be estimated. As far as
I am aware, the polychoric correlations can be estimated using maximum
likelihood. Afterwards, the model parameters can be estimated using a weighted
least squares approach. The ML and DWLS fit functions are defined as follows:

\begin{align}
  F_{ML} &= \ln |S_{ML}| - \ln |\Sigma| + \text{trace}[(S_{ML})(\Sigma^{-1})] - p \tag{ML fit function} \\
  F_{DWLS} &= [S_{DWLS}-\Sigma]' W^{-1}_D [S_{DWLS}-\Sigma].                      \tag{DWLS fit function}
\end{align}

$S_{ML}$ is the covariance or correlation matrix and $S_{DWLS}$ contains the
polychoric correlations. $\Sigma$ is the reproduced covariance or correlation
matrix and depends on the models parameters. $W_D^{-1}$ is a diagonal weight
matrix, with weights that are inversely proportional to the variances of the
polychoric correlations (\cite{yangwallentin2010}). We can therefore conclude
that both approaches aim at minimizing the difference between a reproduced and
sample covariance or correlation matrix. An important difference, however, is
that the least squares approach allows a weighting for correcting the bias that
is present in the MLE approach.

\begin{minipage}{\linewidth}
\begin{lstlisting}
model <- "
    # measurement
    depression         =~ C1PA63 + C1PA64 + C1PA65 + C1PA66 + C1PA67 + C1PA68 + C1PA69
    harm_avoidance     =~ C1SE7V + C1SE7D + C1SE8 + C1SE9
    self_directedness  =~ C1SE14O + C1SE14P + C1SE14R
    social_functioning =~ C1SE1BB + C1SE1D + C1SE1HH + C1SE1J + C1SE1P + C1SE1V

    # structural
    social_functioning ~ harm_avoidance + a*self_directedness
    depression         ~ b*social_functioning + c*self_directedness
    
    IE := a*b
    TE := c + (a*b) 
"

fit <- sem(sem, data=data, ordered=TRUE, meanstructure=FALSE,
           estimator="DWLS")
summary(sem, standardized=TRUE, fit.measures=TRUE)
modindices(sem, sort=TRUE, maximum.number=20)
\end{lstlisting}
\end{minipage}

\FloatBarrier\subsection{Invariance testing}

% Explain the drop in df

Before continuing any further, it is important to establish the measurement
properties of the model. Test bias, which occurs when items are not measuring
the underlying constructs in the same way across groups, should be avoided at
all costs. By simultaneously putting restrictions on multiple parameters of the
measurement model, such equivalence can be tested (\cite{brown2015}). From the
previous section it is clear that the last two waves of the MIDUS dataset have
been combined. Are the changes in a construct then due to changes in the
construct itself or due to changes in how the construct is measured over time?
Moreover, another source of test bias, the sex of the participants, will be
tested as well.

\begin{minipage}{\linewidth}
\begin{lstlisting}
cfa.model <- "
    depression         =~ C1PA63 + C1PA64 + C1PA65 + C1PA66 + C1PA67 + C1PA68 + C1PA69
    harm_avoidance     =~ C1SE7V + C1SE7D + C1SE8 + C1SE9
    self_directedness  =~ C1SE14O + C1SE14P + C1SE14R
    social_functioning =~ C1SE1BB + C1SE1D + C1SE1HH + C1SE1J + C1SE1P + C1SE1V
"
cfa.fit.conf.sex     <- cfa(cfa.model, data=data, ordered=TRUE, estimator="DWLS",
                        group="C1PRSEX", meanstructure=TRUE)
cfa.fit.tau.sex      <- cfa(cfa.model, data=data, ordered=TRUE, estimator="DWLS",
                       group="C1PRSEX", group.equal=c("loadings"), meanstructure=TRUE)
cfa.fit.parallel.sex <- cfa(cfa.model, data=data, ordered=TRUE, estimator="DWLS",
                            group="C1PRSEX", group.equal=c("loadings", "intercepts"),
                            meanstructure=TRUE)

lavTestLRT(cfa.fit.conf.sex, cfa.fit.tau.sex)
lavTestLRT(cfa.fit.tau.sex, cfa.fit.parallel.sex)
summary(cfa.fit.conf.sex, fit.measures=TRUE, standardized=TRUE)\$fit
summary(cfa.fit.tau.sex, fit.measures=TRUE, standardized=TRUE)\$fit
summary(cfa.fit.parallel.sex, fit.measures=TRUE, standardized=TRUE)\$fit
\end{lstlisting}
\end{minipage}

Configural invariance is the first step in this process and indicates that the
factor structure is equal across groups. Second, metric invariance is tested by
constraining the factor loadings to be equal. Lastly, by also
constraining the indicator intercepts to be equal, scalar invariance can be
evaluated. A stepwise procedure can be employed by beginning with the least
restricted solution and gradually testing whether the models $\chi^2$ are
significantly different from each other. Essentially, a likelihood ratio test is
used: a rejection of the null hypothesis then indicates that invariance cannot
be concluded and that the more restricted model has a worse fit
(\cite{brown2015}). However, it should be noted that the $\chi^2$ test is
sensitive to sample size. Due to the large sample size in this study, other fit
indices (RMSEA, SRMR, CFI and TLI) will therefore be used as well. Based on my
observation, a drop of at most 0.01 in CLI or TLIl results in trivial model fit
and measurement invariance can be concluded (\cite{chan2020}). I have not found
guidelines for RMSEA and SRMR, but they can still be used informally to make an
evaluation. The calculation and interpretation of the fit indices will be
discussed in more detail later on.

\begin{table}[h]
\captionsetup{singlelinecheck=off}
\caption{Sex measurement invariance}
\label{tab:sex_invariance}
\scalebox{0.9}{
\begin{tabular}{|l|l|l|l|l|l|l|l|l|l|}
\hline
\textbf{Invariance form} & $\mathbf{\chi^2}$ & \textbf{df} & $\mathbf{\chi^2}$ \textbf{diff.} & \textbf{df diff.} & \textbf{p-value} & \textbf{RMSEA} & \textbf{SRMR} & \textbf{CFI} & \textbf{TLI} \\ \hline
Configural               & 651.13            & 328         & /                                & /                & /                & 0.057          & 0.090         & 0.956        & 0.949         \\ \hline
Metric                   & 701.56            & 344         & 50.44                            & 16               & \textless{}0.001 & 0.059          & 0.092         & 0.952        & 0.947         \\ \hline
Scalar                   & 726.79            & 380         & 25.23                            & 36               & 0.910            & 0.055          & 0.092         & 0.953        & 0.953         \\ \hline
\end{tabular}
}
\end{table}

First, the invariance of the model with respect to sex will be tested. It
should be noted that there are 168 males and 440 females in the sample. The
female group therefore has a higher contribution to the $\chi^2$ and can
contribute more to the model misfit. The $\chi^2$ difference between the
configural and metric model of 50.44 is itself $\chi^2$ distributed with 16
degrees of freedom. In the metric model, there are 16 parameters less to
estimate, since there are 16 factor loadings which are constrained to be equal
across the male and female groups. Unfortunately, the p-value associated with
the $\chi^2$ test is highly significant (p<0.001), which indicates that the
configural model has a better fit than the metric model. Hence, metric
invariance cannot be concluded based on this test. Next, the $\chi^2$ difference
between the metric and scalar model of 25.23 is not significant (p=0.910). Based
on this test, scalar invariance cannot be concluded, since metric invariance is
still a necessary prerequisite. However, other fit measures should be taken into
account due to the large sample size. The RMSEA, CFI and TLI directly correct
for complexity through the degrees of freedom, while the SRMR indirectly makes a
correction through the dimensionality of the reproduced covariance (correlation)
matrix. It is therefore reasonable to expect the fit measures to be more or less
the same for the configural, metric and scalar models if the invariance holds.
Indeed, as shown in Table \ref{tab:sex_invariance} this is the case. I will
therefore conclude that scalar invariance for the male and female groups holds.

\begin{minipage}{\linewidth}
\begin{lstlisting}
cfa.fit.conf.wave     <- cfa(cfa.model, data=data, ordered=TRUE, estimator="DWLS",
                         group="source")
cfa.fit.tau.wave      <- cfa(cfa.model, data=data, ordered=TRUE, estimator="DWLS",
                        group="source", group.equal=c("loadings"))
cfa.fit.parallel.wave <- cfa(cfa.model, data=data, ordered=TRUE, estimator="DWLS",
                             group="source", group.equal=c("loadings", "intercepts"))

lavTestLRT(cfa.fit.conf.wave, cfa.fit.tau.wave)
lavTestLRT(cfa.fit.tau.wave, cfa.fit.parallel.wave)
summary(cfa.fit.conf.wave, fit.measures=TRUE, standardized=TRUE)\$fit
summary(cfa.fit.tau.wave, fit.measures=TRUE, standardized=TRUE)\$fit
summary(cfa.fit.parallel.wave, fit.measures=TRUE, standardized=TRUE)\$fit
\end{lstlisting}
\end{minipage}


\begin{table}[h]
\captionsetup{singlelinecheck=off}
\caption{Time measurement invariance}
\label{tab:time_invariance}
\scalebox{0.9}{
\begin{tabular}{|l|l|l|l|l|l|l|l|l|l|}
\hline
\textbf{Invariance form} & $\mathbf{\chi^2}$ & \textbf{df} & $\mathbf{\chi^2}$ \textbf{diff.} & \textbf{df diff.} & \textbf{p-value} & \textbf{RMSEA} & \textbf{SRMR} & \textbf{CFI} & \textbf{TLI} \\ \hline
Configural               & 653.66            & 328         & /                                & /                 & /                & 0.057          & 0.092         & 0.956        & 0.949        \\ \hline
Metric                   & 688.60            & 344         & 34.94                            & 16                & 0.004            & 0.057          & 0.094         & 0.954        & 0.949        \\ \hline
Scalar                   & 702.82            & 380         & 14.22                            & 36                & 0.999            & 0.053          & 0.093         & 0.957        & 0.957        \\ \hline
\end{tabular}
}
\end{table}

Second, the two latest waves of the MIDUS dataset have been combined. The second
wave was collected in 2004 and 2005, while the third wave originates from 2013
and 2014. This source of invariance should be investigated as well, because we
want to be sure that changes in a construct are due to the construct itself
changing and not because the measurement of the construct has changed over time.
Again, we have to conclude that there is no evidence for metric and scalar
invariance based on the $\chi^2$ test. However, we now know that attention
should be paid to other fit measures due to the large sample size. As shown in
Table \ref{tab:time_invariance}, the RMSEA, SRMR, CFI and TLI are very similar
for the configural, metric and scalar models. In fact, the  CFI and TLI of the
model with scalar invariance are higher. I will therefore conclude that scalar
invariance for the two waves hold, but it should be noted that the result of the
$\chi^2$ test is not significant.

\FloatBarrier\subsection{Measurement model}

% Measurement model
% More interpretation of loadings here
Second, we will take a closer look at the measurement model, which specifies
how indicators relate to their latent constructs. The factor loading can be
interpreted as the regression slope for predicting the indicator from the latent
variable (\cite{brown2015}). The standardized loading is often more interesting,
since it can be interpreted as a correlation and one does not need to worry
about the scale of the variables. By squaring the standardized loading the
communality can be obtained, which indicates the proportion of the variance in
the indicator that is explained by the latent variable. The residual variance
indicates the proportion of the variance that is not explained by the latent
factor and therefore plays a pivotal role as well. Although there are no hard
rules, a popular cut-off value for the communality appears to be 0.5
(\cite{hair2010}). Hence, more than half of the variance in the indicator should
be explained by the latent variable. Based on my observation, communalities that
are a little bit lower are also acceptable, as long as there is a good
theoretical justification for the relationship between the factor and indicator.
The standardized loading should then be larger than 0.7, which means that there
is a high correlation and the indicator does a good job at reflecting the latent
construct.

%     OK PA63 During those two weeks, did you lose interest in most things?
% NOT OK PA64 Thinking about these same two weeks, did you feel more tired
%             out or low on energy?
% NOT OK PA65 During those same two weeks, did you lose appetite?
% NOT OK PA66 Did you have more trouble falling asleep than you usually do
%             during those two weeks?
% NOT OK PA67 During that same two week period, did you have a lot more
%             trouble concentrating than usual?
%     OK PA68 People sometimes feel down on themselves, no good, or worthless.
%             During that two-week period, did you feel this way?
% NOT OK PA69 Did you think a lot about death - either your own, someone else's
%             or death in general - during those two weeks?

Inspecting Table \ref{tab:measurement_base}, it is evident to see that the
measurement model is lacking in some places. Especially the indicators that are
associated with depression are problematic.
Using PA63, the respondent was asked about losing interest in most things. The
high standardized loading of 0.850 indicates that the indicator is strongly
correlated with its construct. The same applies to PA68 (participant feels down,
no good or worthless), which has a communality of 0.572. In other words, 57.2\%
of the variance in this indicator is explained by the latent variable.
Unfortunately, we have to conclude that the other depression indicators have a
standardized loading that is too low. Consequently, they are weakly correlated
with their construct and have a residual variance that is too high. 
The variables PA64, PA65, PA66, PA67 and PA69 are the problematic cases. In
fact, the loading associated with PA66 is not even significantly different from
zero (p = 0.136). The indicators PA64, PA65 and PA66 assess feeling low on
energy, a loss of appetite and trouble falling asleep. PA67 and PA69 evaluate
trouble concentrating and often thinking about death. A simple way to improve
the model fit may be to reduce the number of variables that load on depression
by deleting these problematic indicators. However, this action would lead to a
decline of the theoretical support and validity of the model as well since these
variables were specifically designed by the authors of the dataset to load on
depression (\cite{hair2010}).

\begin{table}[h!]
\captionsetup{singlelinecheck=off}
\caption{Measurement model}
\label{tab:measurement_base}
\scalebox{0.8}{
\begin{tabular}{|l|l|l|l|l|l|l|l|}
\hline
\textbf{Variable} & \textbf{Loading} & \textbf{Standard error} & \textbf{z-value} & \textbf{p-value}     & \textbf{St. loading}           & \textbf{Communality} & \textbf{Unique var.}   \\ \hline
PA63              & 1.000            &                         &                  &                      & 0.850 \hfill ($\lambda_{11}$)  &  0.722               & 0.278 \hfill ($\delta_{11}$)  \\ \hline
PA64              & 0.631            & 0.073                   &  8.662           & \textless 0.001      & 0.536 \hfill ($\lambda_{12}$)  &  0.287               & 0.713 \hfill ($\delta_{12}$)  \\ \hline
PA65              & 0.198            & 0.047                   &  4.236           & \textless 0.001      & 0.168 \hfill ($\lambda_{13}$)  &  0.028               & 0.972 \hfill ($\delta_{13}$)  \\ \hline
PA66              & 0.071            & 0.048                   &  1.492           & 0.136                & 0.061 \hfill ($\lambda_{14}$)  &  0.004               & 0.996 \hfill ($\delta_{14}$)  \\ \hline
PA67              & 0.646            & 0.066                   &  9.757           & \textless 0.001      & 0.548 \hfill ($\lambda_{15}$)  &  0.301               & 0.699 \hfill ($\delta_{15}$)  \\ \hline
PA68              & 0.890            & 0.078                   & 11.460           & \textless 0.001      & 0.757 \hfill ($\lambda_{16}$)  &  0.572               & 0.428 \hfill ($\delta_{16}$)  \\ \hline
PA69              & 0.360            & 0.051                   &  7.005           & \textless 0.001      & 0.306 \hfill ($\lambda_{17}$)  &  0.094               & 0.906 \hfill ($\delta_{17}$)  \\ \hline
%                                                                                                                                
SE7V              & 1.000            &                         &                  &                      & 0.602 \hfill ($\lambda_{21}$)  &  0.362               & 0.637 \hfill ($\delta_{21}$)  \\ \hline
SE7D              & 1.203            & 0.160                   &  7.504           & \textless 0.001      & 0.724 \hfill ($\lambda_{22}$)  &  0.525               & 0.475 \hfill ($\delta_{22}$)  \\ \hline
SE8               & 1.194            & 0.155                   &  7.718           & \textless 0.001      & 0.719 \hfill ($\lambda_{23}$)  &  0.518               & 0.482 \hfill ($\delta_{23}$)  \\ \hline
SE9               & 0.740            & 0.106                   &  7.004           & \textless 0.001      & 0.446 \hfill ($\lambda_{24}$)  &  0.199               & 0.801 \hfill ($\delta_{24}$)  \\ \hline
%                                                                                                                                
SE14O             & 1.000            &                         &                  &                      & 0.821 \hfill ($\lambda_{31}$)  &  0.675               & 0.325 \hfill ($\delta_{31}$)  \\ \hline
SE14P             & 0.982            & 0.049                   & 19.988           & \textless 0.001      & 0.807 \hfill ($\lambda_{32}$)  &  0.651               & 0.349 \hfill ($\delta_{32}$)  \\ \hline
SE14R             & 0.851            & 0.043                   & 19.704           & \textless 0.001      & 0.699 \hfill ($\lambda_{33}$)  &  0.488               & 0.512 \hfill ($\delta_{33}$)  \\ \hline
%                                                                                                                                
SE1BB             & 1.000            &                         &                  &                      & 0.564 \hfill ($\lambda_{41}$)  &  0.318               & 0.682 \hfill ($\delta_{41}$)  \\ \hline
SE1D              & 0.956            & 0.056                   & 17.145           & \textless 0.001      & 0.539 \hfill ($\lambda_{42}$)  &  0.291               & 0.709 \hfill ($\delta_{42}$)  \\ \hline
SE1HH             & 1.398            & 0.066                   & 21.183           & \textless 0.001      & 0.788 \hfill ($\lambda_{43}$)  &  0.621               & 0.379 \hfill ($\delta_{43}$)  \\ \hline
SE1J              & 1.345            & 0.064                   & 21.088           & \textless 0.001      & 0.759 \hfill ($\lambda_{44}$)  &  0.575               & 0.425 \hfill ($\delta_{44}$)  \\ \hline
SE1P              & 1.323            & 0.064                   & 20.568           & \textless 0.001      & 0.746 \hfill ($\lambda_{46}$)  &  0.556               & 0.444 \hfill ($\delta_{46}$)  \\ \hline
SE1V              & 1.113            & 0.060                   & 18.606           & \textless 0.001      & 0.628 \hfill ($\lambda_{47}$)  &  0.394               & 0.606 \hfill ($\delta_{47}$)  \\ \hline
\end{tabular}                                                                                                                                           
}
\end{table}

% SE7D It might be fun and exciting to be in an earthquake.
% SE7V It might be fun learning to walk a tightrope.
% SE8  Of these two situations, I would dislike more: Situation 1:
%      Riding a long stretch of rapids in a canoe; Situation 2:
%      Waiting for someone who's late.
% SE9  Of these two situations, I would dislike more: Situation 1:
%      Being at the circus when two lions suddenly get loose
%      down in the ring; Situation 2: Bringing my whole family
%      to the circus and then not being able to get in because a
%      clerk sold me tickets for the wrong night.
%      => not really a choice between a harm and no harm situation, but between
%         a harm and embarrassment situation

Next, the harm avoidance construct, which measures whether a behaviour is done
to avoid novelty and punishment, plays a central role in this work. It is
measured using four indicators: SE7V, SE7D, SE8 and SE9. In SE7V the participant
is asked whether or not it might be fun to experience an earthquake. By squaring
the standardized loading of 0.602, a somewhat low communality of 0.362 is
obtained. In other words, there is a moderate correlation between the indicator
and the harm avoidance construct, but the residual variance is too high. The
variable SE7D was formed by asking the interviewee whether or not it might be
fun to learn to walk the tightrope. The standardized loading of 0.724 is high enough.
Next, SE8 and SE9 are used to measure harm avoidance by presenting the respondent
with two situations. Afterwards, a choice has to be made between them. The
standardized loadings are, respectively, 0.729 and 0.446. In the first question
the harmful situation is riding a long stretch of rapids in a canoe, while the
safe situation is waiting for someone who is late. In the second question the
harmful situation is being at the circus when two lions get loose. The safe
situation is bringing the family to the circus, but not being able to get in.
The last question therefore not really presents a choice between a harmful and
a safe situation, but rather a choice between a harmful and an embarrassing
situation. Perhaps this could be the reason why the standardized loading is
lower than in SE9.

% SE14O I like to make plans for the future.
% SE14R I know what I want out of life.
% SE14P I find it helpful to set goals for the near future.

Third, self-directedness has been described as a form of self-determination and
ability to regulate behaviour to suit goals and values. Moreover, it should be
related to the harm avoidance construct (\cite{tse2011}).
The variables SE14O, SE14R and SE14P are used to measure this latent variable.
SE14O evaluates whether or not the respondent likes to make plans for the future.
The standardized loading of 0.821 indicates that there is a high correlation
between the indicator and its latent variable. SE14R is used to measure whether
the participant knows what he or she wants out of life. Again, the standardized
loading of 0.807 is high enough. Lastly, SE14P is used to evaluate whether the
participant finds it helpful to set goals for the near future. The standardized
loading of 0.699 is a bit lower, but still high enough nonetheless.

\begin{figure}[h]
\centering
\includegraphics[width=14cm]{../visualizations/base_model.png}
\caption{Summary of the (standardized) base model. hr\_: harm avoidance,
         sl\_: self-directedness, sc\_: social functioning, dpr: depression}
\end{figure}

% SE1BB People would describe me as a giving person, willing to share my time
%       with others.
% SE1D  Most people see me as loving and affectionate.
% SE1HH I have not experienced many warm and trusting relationships with others.
% SE1J  Maintaining close relationships has been difficult and frustrating for me.
% SE1I  I think it is important to have new experiences that challenge how you think
%       about yourself and the world.
% SE1P  I often feel lonely because I have few close friends with whom to share my
%       concerns.
% SE1V  I enjoy personal and mutual conversations with family members and friends.

Lastly, social functioning plays an important role, since it is believed to have
a direct effect on depression (\cite{tse2011}). Seven indicators have been used
to measure this latent variable.
Using SE1BB the respondent was presented with a self-evaluation statement:
`People would describe me as a giving person, willing to share my time with
others.' The standardized loading of 0.564 indicates a poor correlation and a
high residual variance.
The same problem is present in SE1D, which has a standardized loading of 0.539.
SE1HH is used to measure whether the respondent has experienced many warm and
trusting relationships with others. The standardized loading of 0.788 indicates that there
is a high correlation and the indicator does a good job in explaining the variance
of the latent variable.
In SE1J it was asked whether the interviewee has experienced difficulties in
maintaining close relationships. The standardized loading of 0.759 is high enough
and indicates that a one standardized unit increase in social functioning is
estimated to lead to an increase of 0.759 standardized units in SE1J.
Next, SE1P is used to measure whether the respondent often feels lonely because
he or she has few close friends with whom to share his or her concerns. The
standardized loading of 0.746 is high enough.
SE1V is used to evaluate whether the participant enjoys personal and
mutual conversations with family members and friends. The loading is borderline
(0.628), but personally I would still consider it high enough.

To sum up, there are a fair amount of indicators that have a low standardized
loading. Some authors would therefore delete them from the dataset and refit the
model. However, I have decided to leave them in the model, since there is a good
theoretical justification for their relationship with the latent variable. The
variables were specifically designed by the authors of the dataset to load on
certain latent variables. Furthermore, one needs to take this problem into
account when interpreting the results from the structural model and model
evaluation. A chain is only as strong as its weakest link and in a structural
equation model the measurement model is a very important link.

\FloatBarrier
\subsection{Structural model}

% Structural model
The structural model is of great interest in this work, since it allows us
to make conclusions about the relationships between the constructs.
First, we may consider the direct effect of harm avoidance ($\beta_1$) and
self-directedness ($\beta_2$) on social functioning. Both parameters estimates
are highly significant (p<0.001) and indicate a negative relationship with
social functioning. On the one hand, it is estimated that there is a weak,
negative relationship between harm avoidance and social functioning. The
standardized coefficient of -0.172 indicates that an increase of one
standardized unit in harm avoidance is estimated to lead to a decrease of 0.172
standardized units in social functioning. On the other hand, it appears that
there is a stronger relationship between self-directedness and social
functioning. This result is a bit counterintuitive, since the standardized
parameter of -0.548 indicates a strong negative relationship. I would have
personally expected a positive relationship between the two constructs.

Following the theory proposed by \textcite{tse2011}, we may expect there to be
a significant effect of social functioning on depression. Indeed, the results
indicate that there is a negative pattern between the two in the sample, and
this can be generalized to the population (p < 0.001). However, the standardized
coefficient of 0.206 indicates a positive relationship, which is counterintuitive.
In the literature, it is often stated that there is a negative relationship
between the two (\cite{tse2011}).
Lastly, we may consider the direct and indirect effect of self-directedness on
depression. The standardized coefficient of -0.427 indicates a strong negative
direct relationship between the two (p < 0.001). In other words, more
self-directedness is estimated to lead to less depression.
Next, the indirect effect of -0.075 is calculated by multiplying the
standardized parameter estimates of the direct effect of self-directedness on
social functioning ($\beta_2 = -0.548$) and the direct effect of social
functioning on depression ($\beta_3 = 0.136$). 
The total effect (-0.502) can be calculated by adding the direct effect
($\beta_4 = -0.427$) to the indirect effect. The indirect, direct and total
effects are all highly significant (p < 0.001).

\begin{table}[h]
\vspace{0.2cm}
\begin{equation}
\footnotesize
  \begin{cases}
    \textrm{social functioning} & = \beta_1 \textrm{harm avoidance} + \beta_2 \textrm{self-directedness} + \delta_1 \\
    \textrm{depression} & = \beta_3 \textrm{social functioning} + \beta_4 \textrm{self-directedness} + \delta_3
    \tag{Structural model}
  \end{cases}
\end{equation}
\vspace{0.2cm}
\captionsetup{singlelinecheck=off}\caption{Structural model}
\label{tab:base_structural}
\scalebox{0.8}{
\begin{tabular}{|l|l|l|l|l|l|}
\hline
\textbf{Parameter} & \textbf{Coefficient} & \textbf{Stand. coefficient} & \textbf{Standard error} & \textbf{z-value} & \textbf{p-value} \\ \hline
$\beta_1$          &  -0.161 \hfill       & -0.172                      & 0.033                   &  -4.824          & \textless 0.001  \\ \hline
$\beta_2$          &  -0.376 \hfill       & -0.548                      & 0.024                   &  -15.948         & \textless 0.001  \\ \hline
$\beta_3$          &   0.206 \hfill       &  0.136                      & 0.064                   &   3.214          & 0.001            \\ \hline
$\beta_4$          &  -0.442 \hfill       & -0.427                      & 0.056                   &  -7.956          & \textless 0.001  \\ \hline
\end{tabular}
}
\end{table}

Moreover, the latent variables social functioning and depression have a residual
variance, since they are not exogenous in nature. In the standardized solution,
the residual variance indicates the portion of the variance that is not
accounted for by the latent variable. Both depression and social functioning
have a high residual variance: 0.737 and 0.690, respectively. Although we don't
have anything to compare these numbers to, this insight still indicates that
something is not entirely correct with the model. I would have expected the
residual covariances to be lower if the latent variables were being predicted
correctly.

Strongly related to the structural model is the notion of discriminant validity.
Discriminant validity gives an indication that theoretically different
constructs should not be highly intercorrelated. In other words, if two latent
variables are highly correlated they could represent the same construct and they
could be merged into one latent variable to obtain a more parsimonious solution
(\cite{brown2015}). The low and insignificant (p=0.108) standardized covariance
or correlation of -0.1 between harm avoidance and self-directedness indicates
that there is little evidence for poor discriminant validity.

\pagebreak\subsection{Goodness of fit}

Fourth, the goodness of fit of the model will be evaluated using $\chi^2$, SRMR,
RMSEA, CFI and TLI. Afterwards, the sources of misfit will be further
investigated using modification indices. It was previously concluded that there
are some problems with the measurement model. The structural model seemed fine,
but the results were a bit counterintuitive. This evaluation step is therefore
crucial to further determine if the model is a good fit for the data or not.

\subsubsection{Test statistics}

% Goodness of fit
The $\chi^2$ statistic is closely related to the fit of the model and is very
popular in the literature, but it has received some important criticisms.
It has been noted that it is inflated by sample size and in many instances the
underlying distribution is not $\chi^2$ distributed (\cite{brown2015}).
In this illustration, the test statistic of 505.23 is larger than the critical
value of 105.52. Hence, the null hypothesis that this model is equal to a
perfectly fitting model can be rejected and poor model fit is concluded.
However, given the large sample size this result should not be trusted.

% Absolute fit
Absolute fit indices have therefore been employed. They are absolute in the
sense they assess the quality of the solution without taking into account model
parsimony. First, the standardized root mean square residual (SRMR) can be
interpreted as the square root of the average standardized residual covariance
(polychoric correlation). It can be calculated using the following equation,
where $p$ is the number of indicators and $\epsilon$ is the vector of the
standardized residual covariances (\cite{shi2020}). In this illustration a SRMR
of 0.081 was obtained, which indicates borderline poor model fit as it is just
above the target of 0.08.

\begin{minipage}{0.48\linewidth}
\begin{equation}
\label{eq:srmr}
  SRMR = \sqrt{\dfrac{\epsilon \epsilon}{p(p+1)/2}}
\end{equation}
\end{minipage}
\begin{minipage}{0.48\linewidth}
\begin{equation}
\label{eq:rmsea}
  RMSEA = \sqrt{\dfrac{\chi^2 - df}{N \times df}}
\end{equation}
\end{minipage}

Second, the root mean square error of approximation (RMSEA) is based on the
$\chi^2$ statistic and takes into account the error of approximation in the
population. Interestingly, it is the only fit measure that takes into account
the sample size $N$. The RMSEA takes values between zero and one and the fit of
the model is deemed acceptable if it falls under 0.05. A borderline unacceptable 
fit is obtained with a RMSEA of 0.058.

% Comparative fit
The CFI and TLI are two comparative fit indices that will be evaluated as well.
This group of statistics is called comparative, since they make a comparison
between a restricted null model and an alternative model supplied by
the model-builder (\cite{brown2015}). The comparative fit index (CFI) and
Tucker-Lewis index (TLI) have been shown below. Both measures have a range of
possible values from zero to one and make a correction for complexity through
the degrees of freedom. Values that are close to one imply a good model fit,
since the alternative and null model will then be close to each other.
Generally, 0.9 is taken as a target value. In this work the CFI and TLI
are, respectively, 0.952 and 0.945. To sum up, the fit of this model is
borderline good or bad, depending on which fit measures are taken into account.

\begin{minipage}{0.48\linewidth}
\begin{equation}
\label{eq:cfi}
    CFI = \dfrac{(\chi^2 - df)_{null} - (\chi^2 - df)_{alternative}}{(\chi^2 - df)_{null}}
\end{equation}
\end{minipage}
\begin{minipage}{0.48\linewidth}
\begin{equation}
\label{eq:tli}
    TLI = \dfrac{(\chi^2 / df)_{null} - (\chi^2 / df)_{alternative}}{(\chi^2 / df)_{null}}
\end{equation}
\end{minipage}

\begin{table}[h!]
\captionsetup{singlelinecheck=off}
\caption{Test statistics}
\scalebox{0.9}{
\begin{tabular}{|l|l|l|}
\hline
\textbf{Statistic} & \textbf{Value} & \textbf{Target}   \\ \hline
$\chi^2$           & 505.23         & < 105.52          \\ \hline
CFI                & 0.952          & > 0.9             \\ \hline
TLI                & 0.945          & > 0.9             \\ \hline
RMSEA              & 0.058          & < 0.05            \\ \hline
SRMR               & 0.081          & < 0.08            \\ \hline
\end{tabular}
}
\end{table}

\begin{table}[h!]
\captionsetup{singlelinecheck=off}
\caption{The 10 highest modification indices}
\label{tab:base_modification}
\scalebox{0.83}{
\begin{tabular}{|l|l|l|l|l|l|}
\hline
\textbf{Left hand side} & \textbf{Operation} & \textbf{Right hand side} & \textbf{Modification index} & \textbf{\begin{tabular}[c]{@{}l@{}}Expected parameter\\ change\end{tabular}} & \textbf{\begin{tabular}[c]{@{}l@{}}Stand. expected\\ parameter change\end{tabular}} \\ \hline
SE1BB                   & correlation        & SE1D                     & 90.619                      &  0.329                                                                       &  0.473 \\ \hline
SE14O                   & correlation        & SE14R                    & 44.130                      &  0.317                                                                       &  0.778 \\ \hline
social\_functioning     & loading            & SE14P                    & 36.950                      & -0.551                                                                       & -0.311 \\ \hline
SE1BB                   & correlation        & SE1HH                    & 34.474                      & -0.290                                                                       & -0.570 \\ \hline
social\_functioning     & loading            & PA65                     & 29.688                      & -0.415                                                                       & -0.234 \\ \hline
social\_functioning     & loading            & PA66                     & 26.101                      & -0.403                                                                       & -0.227 \\ \hline
social\_functioning     & loading            & SE14O                    & 22.440                      &  0.463                                                                       &  0.261 \\ \hline
depression              & loading            & SE1D                     & 20.018                      & -0.253                                                                       & -0.215 \\ \hline
social\_functioning     & loading            & SE7V                     & 19.761                      &  0.255                                                                       &  0.144 \\ \hline
SE1HH                   & correlation        & SE1J                     & 19.647                      &  0.170                                                                       &  0.424 \\ \hline
\end{tabular}                                                                                                    
}
\end{table}

\subsubsection{Modification indices}

The modification indices can be used to more precisely investigate sources of
model misfit. They can be calculated for each fixed and constrained parameter in
the model and indicate how much the model $\chi^2$ would drop if a certain
parameter were to be freely estimated. A good fitting model should then also
produce modification indices that are small in magnitude. A modification index
that is greater than 3.84 indicates that the model fit can be significantly
improved if the parameter is freely estimated (\cite{brown2015}). Unfortunately,
the summary shown in Table \ref{tab:base_modification} indicates that there are
various sources of badness of fit associated with the measurement model.

% The
% question is now how to proceed. Since structural equation modeling is at heart
% a theory testing framework, it is important to first investigate the reason why
% a certain modification index is too large. A proper theoretical justification
% needs to be present before freeing a parameter.

First, we may notice three very high modification indicates that have to do with
correlated error terms.
A huge drop in the model's $\chi^2$ of 90.62 can be realized by allowing a
correlated error term between the variables SE1BB and S1D, which share a loading
on social functioning. On the one hand, SE1BB assesses whether the respondent
believes other people would describe him/her as a giving person. On the other
hand, SE1D evaluates whether the respondent believes other people see him/her as
loving and affectionate. Personally, I believe that it is very plausible that
these variables are related to one another.
Next, the modification indices indicate that the model fit would improve
dramatically should a correlated error term be allowed between the
self-directedness variables 1SE14O and 1SE14R. In 1SE14O and 1SE14R it is asked
whether the respondent likes to make plans for the future and knows what to want
out of life, respectively.
Additionally, the modification indices indicate that a correlated error term
should be allowed between the variables 1SE1HH and 1SE1BB. Both variables have a
loading on social functioning. 1SE1HH indicates whether the respondent has
experienced many warm and trusting relationships. 1SE1BB assesses whether the
respondent believes he or she would be described by others as a giving person,
willing to share his or her time.

Second, from Table \ref{tab:base_modification} we can see that the other
modification indices are related to cross-loadings.
Specifically, the variable SE14P (self-directedness) is suggested to
load on social functioning. In 1SE14P it is asked whether the interviewee likes
to make plans for the future. Clearly, this variable is more related to
self-directedness than social functioning.
Furthermore, PA65 and PA66 load on depression and assess whether the respondent
has lost appetite during the last two weeks or had trouble falling asleep. The
model's $/chi^2$ would decrease by 29.69 and 26.10 should a cross-loading be
allowed on social functioning. Clearly, this is a spurious relationship that-
should not be allowed.
SE14O loads on social functioning as well, and asks whether the interviewee
likes to make plans for the future. This variable is clearly more related to
self-directedness than social functioning and a cross-loading should therefore
not be allowed.
Lastly, the modification indices suggest that SE1D should have a loading on
depression. SE1D assesses whether the respondent believes other people see
him/her as loving and affectionate.

\FloatBarrier\pagebreak\section{Conclusion}

In this work, I have set out to investigate a theory about depression that has
been proposed by \textcite{tse2011}. The authors tested their theory on a sample
of university students and suggested that depression can be explained by
self-directedness, social functioning and harm avoidance. By testing their
theory on a larger and more representative sample, I have contributed to the
literature on depression. After testing the invariance properties, a structural
equation model was specified and estimated. Unfortunately, we have to recognize
that the fit of the measurement model was not satisfactory. Moreover, the
modification indices suggested that there are various places where the fit
could be improved. Since the structural model is based on the measurement model,
the findings should be taken with a grain of salt. Indeed, the results of the
structural model were a bit counterintuitive and not in line with the findings
of \textcite{tse2011}. In the future, it would be interesting to make
improvements to the measurement model and re-estimate the structural model.

\FloatBarrier\pagebreak\section{References}
\printbibliography[heading=none]

\FloatBarrier\pagebreak\section{Appendices}
\subsection{Code}

Unfortunately, the R code was too extensive too directly include here. I have
therefore made it \href{https://github.com/vandenbroecksebastiaan/SEM/}{available on Github}
(main.R). The data is included as well.

\subsection{(Fully standardized) matrices}

\begin{minipage}{0.35\linewidth}
  \begin{equation*}
    \pmb{\lambda} = 
      \begin{pmatrix}
      0.850           & 0	              & 0	                & 0           \\
      0.536	          & 0	              & 0	                & 0           \\
      0.168	          & 0	              & 0	                & 0           \\
      0.061	          & 0	              & 0	                & 0           \\
      0.548	          & 0	              & 0	                & 0           \\
      0.757	          & 0	              & 0	                & 0           \\
      0.306	          & 0	              & 0	                & 0           \\
      0	              & 0.602	          & 0	                & 0           \\
      0	              & 0.724	          & 0	                & 0           \\
      0	              & 0.719	          & 0	                & 0           \\
      0	              & 0.446	          & 0	                & 0           \\
      0	              & 0	              & 0.821	            & 0           \\
      0	              & 0	              & 0.807	            & 0           \\
      0	              & 0	              & 0.699	            & 0           \\
      0	              & 0	              & 0	                & 0.564       \\
      0	              & 0	              & 0	                & 0.539       \\
      0	              & 0	              & 0	                & 0.788       \\
      0	              & 0	              & 0	                & 0.759       \\
      0	              & 0	              & 0	                & 0.746       \\
      0	              & 0	              & 0	                & 0.628       
      \end{pmatrix}
  \end{equation*}
\end{minipage}

\begin{minipage}{0.76\linewidth}
\begin{align*}
\pmb{\theta} = \textrm{diag}(& 0.278, 0.713, 0.972, 0.996, 0.699, 0.428, 0.906, 0.637 ,0.475, 0.482, \\
                             & 0.801, 0.325, 0.349, 0.512, 0.682, 0.709, 0.379, 0.425, 0.444, 0.606)
\end{align*}
\end{minipage}

\begin{minipage}{0.33\linewidth}
  \begin{equation*}
    \pmb{\psi} = 
      \begin{pmatrix}
      0.737 &  0     &  0     & 0     \\
      0     &  1     &  0     & 0     \\
      0     & -0.107 &  1     & 0     \\
      0     &  0     &  0     & 0.690
      \end{pmatrix}
  \end{equation*}
\end{minipage}

\begin{minipage}{0.34\linewidth}
  \begin{equation*}
    \pmb{\beta} = 
      \begin{pmatrix}
      0 &  0     & -0.427 & 0.136 \\
      0 &  0     &  0     & 0     \\
      0 &  0     &  0     & 0     \\
      0 & -0.172 & -0.548 & 0    
      \end{pmatrix}
  \end{equation*}
\end{minipage}

\end{document}
